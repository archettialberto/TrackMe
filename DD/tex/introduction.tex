\section{Introduction}
\label{sec:intro}
   \subsection{Purpose}

   \subsection{Scope}

   \subsection{Definitions}

    \subsection{Acronyms and abbreviations}

    \subsection{Revision history}

  \label{sec:revhistory}
    \begin{table}[!h]
      \begin{tabular}{|c|l|}
        \hline
        \textbf{Version}                & \textbf{Log} \\ \hline
        \texttt{v.0.0} &
          DD first draft
        \\
 \hline
      \end{tabular}
    \end{table}



\subsection{Reference Documents} 
% Just a draft
  \subsection{Document structure}

    This document uses the IEEE standards for requirement analysis documents \cite{ieee830} as a guideline towards a clear and logical explanation of its contents:
    \begin{itemize}
      \item Section~\ref{sec:intro} gives a brief introduction on the project to be developed.
      \item Section~\ref{sec:arcdes} describes the architectural design
      \item Section~\ref{sec:userdes} provide an overview on how the user interface(s) of your system will look like
      \item Section~\ref{sec:req} explain how the requirements previously defined in the RASD map to the design elements described in this document
      \item Section~\ref{sec:imp} describes how to implement the subcomponents of the system and how to integrate such subcomponents and test the integration.
      \item Section~\ref{sec:effort} lists the work sessions that drove this project's development, ordered by date, as the hour counter of effort spent by each group member
    \end{itemize}


