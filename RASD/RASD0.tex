\documentclass[12pt]{article}
\usepackage[english]{babel}
\usepackage[utf8x]{inputenc}
\usepackage{amsmath}
\usepackage{graphicx}

\def \thisDocVersion {\texttt{v.0.0}}

% TODO finish section 1

% === RASD ===
%The Requirements analysis and specification document (RASD) contains the description of the scenarios, the use cases that describe them, and the models describing requirements and specification for the problem under consideration. You are to use a suitable mix of natural language, UML, and Alloy. UML and Alloy MUST be part of the documentation. You must also show that you used the Alloy tool for analysis, by reporting the models you obtained by using it. Of course, the initial written problem statement we provide suffers from the typical drawbacks of natural language descriptions: it is informal, incomplete, uses different terms for the same concepts, and the like. You may choose to solve the incompleteness and ambiguity as you wish, but be careful to clearly document the choices you make and the corresponding rationale. You will also include in the document information on the number of hours each group member has worked towards the fulfillment of this deadline. As a reference structure for your document, you should refer to the one reported below that is derived from the one suggested by IEEE. Please include in the document information about the effort spent by each group member for completing this document.

% === TrackMe ===
%TrackMe is a company that wants to develop a software-based service allowing third parties to monitor the location and health status of individuals. This service is called Data4Help. The service supports the registration of individuals who, by registering, agree that TrackMe acquires their data (data acquisition can happen through smartwatches or similar devices). Also, it supports the registration of third parties.
%After registration, these third parties can request:
%\begin{itemize}
%  \item Access to the data of some specific individuals (we can assume, for instance, that they know an individual by his/her social security number or fiscal code in Italy). In this case, TrackMe passes the request to the specific individuals who can accept or refuse it
%  \item Access to anonymized data of groups of individuals (for instance, all those living in a certain geographical area, all those of a specific age range, etc.). These requests are handled directly by TrackMe that approves them if it is able to properly anonymize the requested data. For instance, if the third party is asking for data about 10-year-old children living in a certain street in Milano and the number of these children is two, then the third party could be able to derive their identity simply having people monitoring the residents of the street between 8.00 and 9.00 when kids go to school. Then, to avoid this risk and the possibility of a misuse of data, TrackMe will not accept the request. For simplicity, we assume that TrackMe will accept any request for which the number of individuals whose data satisfy the request is higher than 1000
%\end{itemize}
%As soon as a request for data is approved, TrackMe makes the previously saved data available to the third party. Also, it allows the third party to subscribe to new data and to receive them as soon as they are produced.
%Imagine now that, after some time, TrackMe realizes that a good part of its third-party customers wants to use the data acquired through Data4Help to offer a personalized and non-intrusive SOS service to elderly people. Therefore, TrackMe decides to build a new service, called AutomatedSOS, on top of Data4Help. AutomatedSOS monitors the health status of the subscribed customers and, when such parameters are below certain thresholds, sends to the location of the customer an ambulance, guaranteeing a reaction time of less than 5 seconds from the time the parameters are below the threshold.

\begin{document}

\begin{titlepage}
  \newcommand{\HRule}{\rule{\linewidth}{0.5mm}}
  \center
  \includegraphics[width=140pt]{polimi.png}\\[1cm]
  \textsc{\LARGE Politecnico di Milano}\\[1cm]
  \textsc{\Large Master's Degree in \\Computer Science and Engineering}\\[0.5cm]
  \textsc{\large Software Engineering 2}\\[0.5cm]
  \HRule \\[0.4cm]
  { \huge \bfseries TrackMe\\[0.4cm] Requirements Analysis and Specification Document }\\[0.4cm]

  \HRule \\[1cm]
  \begin{minipage}{0.4\textwidth}
  \begin{flushleft} \large
  \emph{Authors}\\
  Alberto \textsc{Archetti}

  Fabio \textsc{Carminati}
  \end{flushleft}
  \end{minipage}
  ~
  \begin{minipage}{0.4\textwidth}
  \begin{flushright} \large
  \emph{Reference professor} \\
  Elisabetta \textsc{Di Nitto}
  \end{flushright}
  \end{minipage}\\[1cm]
  {\large \thisDocVersion\ - \today}\\[1cm]
  \vfill
\end{titlepage}

\clearpage
\pagenumbering{gobble}
\nonumber
\tableofcontents
\clearpage
\pagenumbering{arabic}

\clearpage
\section{Introduction}
\label{sec:intro}

  \subsection{Purpose}

    This is the Requirement Analysis and Specification Document (RASD) of \texttt{Data4Help} and \texttt{AutomatedSOS} services, commissioned by TrackMe company. We will specify goals, domain assumptions, requirements, interfaces and high-level models using \texttt{UML} and \texttt{Alloy} languages of the systems that will be produced. This is an important step in software development, because identifying from the starts the correct scope, the constraints and the overall structure of our products is the key to produce mantainable and secure software that correctly responds to the stakeholder's needs. \\
    The audience of this document is very wide. It includes
    \begin{itemize}
      \item stakeholders, as it acts as a contract that certifies what is required to our final product in order to satisfy their needs
      \item developers that will be guided by this document's prescriptions
      \item testers that are asked to verify the correspondence between the implementation and the requirements
      \item managers, in order to keep track of the project development
    \end{itemize}
    % FABIO ho rimesso questa lista perché nel documento dell'IEEE si dice che qua vanno segnati i destinatari del documento

    Requirement analysis and elicitation is an iterative process. This is the version \thisDocVersion\ of the RASD document. See section~\ref{sec:revhistory} for more details on revision history.

  \subsection{Scope}

      \subsubsection{Project description}

        TrackMe wants to develop a software-based service that allows individual users to collect health data, called \texttt{Data4Help}. This data, stored in the \texttt{Data4Help} system, can be retrived and visualized according to different filters and projections. \\
        The system allows third parties registration. Third parties can request access to users'collected data in two ways:

        \begin{description}
          \item [\textbf{Single-person data}] After the request by the third party is made through the system interface, the system asks the user for authorization; if positively provided, the third party is granted access to the user's data
          \item [\textbf{Amonymous-group data}] Thirs parties can be interested in big amounts of data, regarding who are the people that are providing it; the system, once the request by the third party is sent, checks if the data can be effectively anonymized (it must find at least 1000 people that match the third party request) and, if positively evaluated, grants access to the anonymized data to the third party that requested it
        \end{description}

        Third parties can subscribe to new data and receive it as soon as it is collected by the system.

        Another service that TrackMe wants to develop is \texttt{AutomatedSOS}, built on \texttt{Data4Help}. This service analyzes users'data and calls a SOS whenever data exceedes the basic health parameters. For this particular purpose, system performances will be a critical aspect to be taken into account, because even the slightest delay matters in critical health situations.

      \subsubsection{Goals}

        Here we present the goals that will be reached once the project is completed:

        %G1-Allow the user to share his personal data with Data4Help.\\
        %G1.2-Allow a user to update his DataSet anytime.\\
        %G2-Third parties can access either at specificl DataSet or anonymized DataSets of a group of individuals.\\
        %G2.1-Third parties can subscribe to new data and receive them as soos as they are available.\\
        %G3-Elderly user can subscribe to AutomatedSOS.\\
        %[G5]-User has to know which third party has collected his own data( already in G2!?)
        %registration of user,third party is available requirement:

        %Magari martedì li confermiamo?

        \begin{description}
          % TODO end defining goals
          \item [\texttt{G1}] goal 1
          \begin{description}
            \item [\texttt{G1.1}] goal 1
          \end{description}
          \item [\texttt{G2}] goal 2
        \end{description}

  \subsection{Definitions, acronyms, abbreviations}

    \subsubsection{Definitions}

      \begin{description}
        \item[]
        %\item [Data Set]
        % FABIO prima di threshold ci conviene definire quali parametri raccogliere e cos'è un "set" di parametri per un utente, tipo vettore.OK
        %\item [Threshold] % In the context of \texttt{AutomatedSOS}, it refers to a particular boundary in health parameters space that divides the set of all ...
      \end{description}
      % \textbf{DataSet}=Set of all data shared by the user with TrackMe.\\
      % FABIO ho pensato ad un modo per aggirare il problema della scelta dell'utente di poter cambiare i permessi sulle sue condivisioni: domani ti spiego e ridefiniamo Data Set se ti va bene l'idea.OK
      %\textbf{Threshold}=It refers to a limit for a Health parameter. If crossed implies that the user needs help. \\
      % TODO fill this

    \subsubsection{Acronyms}

      \begin{description}
        \item[]
      \end{description}

    \subsubsection{Abbreviations}

      \begin{description}
        \item[]
      \end{description}

  \subsection{Revision history}
  
  \label{sec:revhistory}
    \begin{table}[!h]
      \begin{tabular}{|c|l|}
        \hline
        \textbf{Version}                & \textbf{Log} \\ \hline
        \texttt{v.0.0} &
          Introduction sketch
        \\ \hline
      \end{tabular}
    \end{table}

  \subsection{Reference documents}

    See References for details on the consulted documents.

  \subsection{Document structure}

    This document uses the IEEE standards for requirement analysis documents as a guideline towards a clear and logical explanation of its contents:
    \begin{itemize}
      \item Section~\ref{sec:intro} gives a brief introduction on the project to be developed and adds notes on references and revisions
      \item Section~\ref{sec:overdesc} describes the world and the shared phenomena, by defining assumptions and constraints; it identifies also the goals and the main functions of the project
      \item Section~\ref{sec:specreq}, as the main part of this document, is about requirement analysis; it has also sections about interfaces of the system and software attributes
      \item Section~\ref{sec:alloy} contains the \texttt{Alloy} model that certifies correctness of goals implication by requirements and domain assumptions
      \item Section~\ref{sec:effort} lists the overall modifications and additions to this document, ordered by date, as the hour counter of effort spent by each group member
    \end{itemize}

\clearpage
\section{Overall description}
\label{sec:overdesc}

  \subsection{Product perspective}
  \subsection{Product functions}
  \subsection{User characteristics}

    \subsubsection{Actors}

      \begin{description}
        %\item \textbf{Visitor}:Someone who hasn't got an account of TrackMe and therefore must complete registration before having granted the access to TrackMe.
        %visitor useful only if we create an adeguate use case (ex. first login)
        % FABIO secondo me possiamo chiamarlo user e basta, perché altrimenti sembra che ci possa essere uno che usa l'applicazione senza essere registrato
        \item [User] Person that has successfully created an account of TrackMe. She or He can exploit all the functionalities of the application
        \item [Third Party] Entity that can request to Data4Help  the access  to either individual or group DataSets
        %maybe split between Data4Help and AutomatedSOS user?
      \end{description}

  \subsection{Assumptions, dependencies, constraints}

    \subsubsection{Domain assumptions}

      \begin{description}
        \item[\texttt{D1}] da 1
        \item[\texttt{D2}] da 2
      \end{description}
      %Ci conviene definirli  assieme ai goal
      %D1-The chosen thresholds are enough to describe user health.\\
      %D2-The Data collected by the external sensors and the GPS location are accurate enough.\\
      %D3-There is at least one ambulance available when It is required by AutomatedSOS.\\


\clearpage
\section{Specific requirements}
\label{sec:specreq}

\subsection{Scenarios:}
%  \subsubsection{Scenario 1:}
%Luke is a 65 years old man that after 40 years of  working at the post office finally got retired.Because of the sedentary nature of his work he is worried that he may suffer a stroke.Therefore Luke downloads AutomatedSOS,create an account and shares with it his age,weight,heartbeat and GPS location. From now if he will unfortunately suffer a stroke the app within 5 seconds will call an ambulance.

%\subsubsection{Scenario 2:}
%The franchise BodySlim is opening a new  gym in Milan near Parco Sempione;the most relevant aspect that characterize it, is the timetable:open 365 days, 8 hours per day. In order to maximize the revenue the management would like to know in which time of the day the potential customers made exercise. In order to have a reliable sample BodySlim create a third party account and asked Data4Help to provide a sample of 2000 individuals with particular characteristics: age between 20 and 60,at least once a week they are running in Parco Sempione(this can be checked with GPS position).Now looking at the most common hours chosen for running by these individuals the manager can easily identify the most profitable hours in which the gym should be opened.

    %\subsubsection{Allow a Third Party to access at a speficy DataSet prior User Authorization}
    %	\subsubsection{Allow a Third Party to access at  group of anonymous DataSets made by at least 1000 individuals}
    %check whether the data fits with the request of the third party
    %	\subsubsection{Activation of AutomatedSOS must be granted in case of need}

  \subsection{External interface requirements}
    \subsubsection{User interfaces}
    \subsubsection{Hardware interfaces}
    \subsubsection{Software interfaces}
    \subsubsection{Communication interfaces}
  \subsection{Functional requirements}
  \subsection{Performance requirements}
  \subsection{Design constraints}
    \subsubsection{Standards compliance}
    \subsubsection{Hardware limitations}
    \subsubsection{Any other constraint}
  \subsection{Software system attributes}
    \subsubsection{Reliability}
    \subsubsection{Availability}
    \subsubsection{Security}
    \subsubsection{Mantainability}
    \subsubsection{Portability}

\clearpage
\section{Formal analysis using Alloy}
\label{sec:alloy}

\clearpage
\section{Effort spent}
\label{sec:effort}

\clearpage
\begin{thebibliography}{9}
  \bibitem{assignment} Mandatory Project Assignment AY 2018-2019

  \bibitem{ieee830} IEEE 830-1993 - IEEE Recommended Practice for Software Requirements Specifications

  \bibitem{ieee29148} ISO/IEC/IEEE 29148 - Systems and software engineering — Life cycle processes — Requirements engineering

  \bibitem{wereabledata} Collection and Processing of Data from Wrist Wearable Devices in Heterogeneous and Multiple-User Scenarios \\
  \texttt{https://www.ncbi.nlm.nih.gov/pmc/articles/PMC5038811/}

  \bibitem{googlefitapi} Google Fit API\\
  \texttt{https://developers.google.com/fit/overview}

  \bibitem{slides}
  Slides of the course by Prof. Di Nitto
  \\\texttt{https://beep.metid.polimi.it/}

  \bibitem{latex_templates}
  \LaTeX templates
  \\\texttt{http://www.latextemplates.com/}
  \\\texttt
  {http://www.overleaf.com/latex/examples/title-page-with-logo/hrskypjpkrpd}

  %\bibitem{latexcompanion}
  %Michel Goossens, Frank Mittelbach, and Alexander Samarin.
  %\textit{The \LaTeX\ Companion}.
  %Addison-Wesley, Reading, Massachusetts, 1993.

  %\bibitem{einstein}
  %Albert Einstein.
  %\textit{Zur Elektrodynamik bewegter K{\"o}rper}. (German)
  %[\textit{On the electrodynamics of moving bodies}].
  %Annalen der Physik, 322(10):891–921, 1905.
\end{thebibliography}

\end{document}
