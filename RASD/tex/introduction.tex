\section{Introduction}
\label{sec:intro}

  \subsection{Purpose}

      \subsubsection{Project description}

        TrackMe wants to develop a software-based service that allows individual users to collect health data, called \texttt{Data4Help}. This data can be retrived from the system and visualized according to different filters by a user interface. \\
        The system allows third parties registration. Third parties can request access to users'collected data in two ways:

        \begin{description}
          \item [\textbf{Single-person data}] After the the third party makes a request to the system for a single user data sharing, by providing  the fiscal code of the user, the system asks the user for authorization; if positively provided, the third party is granted access to the user's data
          \item [\textbf{Amonymous-group data}] Thirs parties can be interested in big amounts of data, but not in who are the people providing it; the system, once the request is sent by the third party, checks if the data can be effectively anonymized (it must find at least 1000 people that can provide data matching the third party request) and, if positively evaluated, grants access to the anonymized data to the third party
        \end{description}

        Third parties can subscribe to new data and receive it as soon as it is collected by the system.

        Another service that TrackMe wants to develop is \texttt{AutomatedSOS}, built on \texttt{Data4Help}. This service analyzes users'data and calls a SOS whenever data exceedes the basic health parameters. For this particular purpose, system performances will be a critical aspect to be taken into account, because even the slightest delay matters in critical health situations.

      \subsubsection{Goals}
      \label{sec:goals}

        Here we present the goals that will be reached once the project is completed:

        \begin{description}
          \item[\texttt{G.U1}] Users can collect, store and manage their health data
          \item[\texttt{G.U2}] Users can choose to have their health monitored; if their health is critical, an ambulance will be dispatched
          \item[\texttt{G.T1}] Third parties can ask single users for their health data sharing
          \item[\texttt{G.T2}] Third parties can request access to anonymized data that comes from groups of people
          \item[\texttt{G.T3}] Third parties can subscribe to new data and receive it as soon as it is produced
        \end{description}

  \subsection{Scope}

    \subsubsection{World}

      Our \textit{world} is composed of two main types of actors: users and third parties. Users are interestes in monitoring their health parameters and third parties are interested in developing services or researches that exploit data gathered from the users. \texttt{Data4Help} is the service that acts as a bridge between these actors'needs. \\
      Phenomena that occur in the \textit{world} and are related to our application domain are
      \begin{itemize}
        \item physical conditions of the users
        \item third parties'projects, researches and interests
        \item SOS system that is able to dispatch ambulances
        \item payment system that processes payments between actors
      \end{itemize}
      These phenomena exist in the \textit{world}, but cannot be observerd directly by our system.

    \subsubsection{Shared phenomena}
    \label{sec:sharedp}

      In order to communicate with the \textit{world}, our system needs to share some aspects with it. We will list the aspects controlled by the world, but observable by the machine:
      \begin{description}
        \item[\texttt{S.1}] physical parameters of the users, gathered through sensors on wereable devices
        \item[\texttt{S.2}] third parties requests to the system for the data they need
        \item[\texttt{S.3}] users'location, acquired through GPS signals
      \end{description}
      On the other hand, the aspects that occur in the machine, but are observable by the world are
      \begin{description}
        \item[\texttt{S.4}] interfaces that organize the gathered data that can be filtered according to time or type of data
        \item[\texttt{S.5}] messages for the SOS system, that are sent in case of critical health of a user
        \item[\texttt{S.6}] payment requests
      \end{description}

  \subsection{Definitions}
  \label{sec:definitions}

    \begin{description}
      \item[Data] Set of values of qualitative or quantitative variables\footnote{https://en.wikipedia.org/wiki/Data}; in the context of \texttt{Data4Help} we will refer to quantitative variables concerning health parameters (Section~\ref{sec:datagathering})
      \begin{description}
        \item[Anonymous data] \textit{data entry} that doesn't contain information about the user from which it was produced; a \textit{data set} is said to be anonymized if it contains only anonymous \textit{data entries} and its cardinality is greater or equal than 1000
        \item[Data entry] Tuple that corresponds to the user's parameters in a particular moment
        \item[Data set] Set of \textit{data entries}; depending on the context, it can identify a set of entries all belonging to a single user or or a set of anonymous entries belonging to more that 1000 users; a \textit{data set}, among all \textit{data} that the system is storing, can be identified and constructed according to the filters of a third party request (Section~\ref{sec:datasharing})
      \end{description}
      \item[Request] Third parties can ask the system for some data sharing through requests; requests are encoded through filling a form; the system, provided that the request is satisfiable, grants the third party access to the requested data (Section~\ref{sec:datasharing})
      \item[Third party] Actor interested in collecting data from a single user or from an anonymous group of users (Section~\ref{sec:usercharacteristics})
      \item[Threshold] Numerical values related to a particular health parameter; they act as boundaries between the domain of critical health status and normal health status
      \item[User] Actor interested in his/her health data collecting and managing; a user can also be interested in automating SOS calls whenever his health status becomes critical (Section~\ref{sec:usercharacteristics})

      % TODO check this reference https://www.selfcare4me.com/en/parameters-what-do-you-want-to-measure/
    \end{description}

  \subsection{Acronyms and abbreviations}

      \begin{description}
        \item[API] Application Programming Interface
        \item[Data] Whenever the context refers to generic groups of \textit{data entries}, the terms \textit{data} and \textit{data set} are interchangeable
        \item[System] Software product that TrackMe wants to develop; can be interchanged with \textit{S2B}
        \item[S2B] Software To Be
      \end{description}

  \subsection{Revision history}

  \label{sec:revhistory}
    \begin{table}[!h]
      \begin{tabular}{|c|l|}
        \hline
        \textbf{Version}                & \textbf{Log} \\ \hline
        \texttt{v.0} &
          First version of RASD completed
        \\ \hline
      \end{tabular}
    \end{table}

  %\subsection{Reference documents} %See References for details on the consulted documents. % TODO decide whether to leave this or delete it

  \subsection{Document structure}

    This document uses the IEEE standards for requirement analysis documents \cite{ieee830} as a guideline towards a clear and logical explanation of its contents:
    \begin{itemize}
      \item Section~\ref{sec:intro} gives a brief introduction on the project to be developed, by describing its goals and the environment in which it will be released; keeps track of the revision history
      \item Section~\ref{sec:overdesc} describes the world and the shared phenomena, by defining assumptions and constraints; it identifies the main functions of the project
      \item Section~\ref{sec:specreq}, as the main part of this document, is about requirement analysis; it has also sections about interfaces of the system and software attributes
      \item Section~\ref{sec:alloy} contains the \texttt{Alloy} model that certifies correctness of goals implication by requirements and domain assumptions
      \item Section~\ref{sec:graph} contains graphics, like UML diagrams, that help for a better understand of the system structure
      \item Section~\ref{sec:effort} lists the work sessions that drove this project's development, ordered by date, as the hour counter of effort spent by each group member
    \end{itemize}
