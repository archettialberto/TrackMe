\section{Specific requirements}
\label{sec:specreq}
%TODO:Decide where put this section
\subsection{Scenarios}
\subsubsection{Scenario 1}
Luke is a 65 years old man that after 40 years of  working at the post office finally got retired.Because of the sedentary nature of his work he is worried that he may suffer a stroke.\\ Therefore last week Luke downloaded the app AutomatedSOS, the system allowed him to create an account only after having filled all the required user information(see Table 1):from now on the system will check Luke's health parameters. Today Luke's health parameters exceede thresholds because of a stroke:the system recognize this critical situation and therefore within 5 seconds an emergency call to SOS System, providing the GPS location and health status feedback of the Luke,will be made by AutomatedSOS.

\subsubsection{Scenario 2}
The franchise BodySlim is opening a new gym in Milan near Parco Sempione;the most relevant aspect that characterize it, is the timetable:open 365 days, 8 hours per day. In order to maximize the revenue the management would like to know in which time of the day the potential customers make exercise. A reliable sample can be obtained from Data4Help, therefore BodySlim create a third party account filling all the required third party information (see Table 1). Then the franchise make an anonymous-group request (see Table 2) for 2000 individuals with particular characteristics: age between 20 and 60,at least once a week they are running in Parco Sempione(this can be checked with GPS position) to Data4Help. The app find in his internal database these 2000 users and therefore the system immediately shares the required information with BodySlim without asking to the users the authorization .Now looking at the most common hours chosen for running by these individuals the manager can easily identify the most profitable hours in which the gym should be opened. 
%Otherwise if the research is not successfully, Data4Help denied the request of BodySlim which has to find another way to obtain a reliable sample. 
%NO ALTERNATIVES:THIS IS AN EXAMPLE/AN APPLICATION

\subsubsection{Scenario 3}
Rose has just discovered that is expecting a daughter. Harold, her gynecologist, needs to keep monitoring bloody ossygen,electrocardiogram and pulse in order to understand if the pregnancy is proceeding well or not. Therefore he creates a third party account of Data4Help and after the system has checked if the required information(see Table 1) has been correctly inserted, he requires these data to the app. According to the single-user request(see Table 2)  policy Data4Help asks to Rose the consensus for sharing these data. Rose,which really care about her daughter's health, immediately gave the authorization to the app that can make them available to her gynecologist. Nine months later the beautiful and healthy Marie is born.
%her motherRose which is worried about her privacy, wouldn't like to share his personal data anymore so decides to unsubscribe to Data4Help. The system after having received the request, delete her Dataset and her account. From now on neither the gynecologist nor Data4Help will be able to access to Rose's data.

\subsubsection{Scenario 4}
Jack is a policeman that has decided to spend his holidays in a spa called "BeautySPA" which allow him to loss 15 kg thanks to fitness activity and a healthy diet.In order to understand if this treatment is efficient over time the spa and Jack decides that user's health parameters like height,weight and body temperature must be monitored. Therefore the policeman and BeautySPA create respectively a user and a third party account of Data4Help and the system ,after checking that all the required information (see Table 1)are properly added, allow them to exploit all Data4Help functionalities. Indeed  BeautySPA immediately apply for the three Jack's health parameters to the app. The app recognize this as a single-user request(see Table 2)  so asks to Jack the authorization whether to provide them to the spa or not. Obviously he accept and therefore Data4Help make available these data. Unfortunately after just 1 month the policeman has already had a weight increase of 6 kg. So in order to better understand which may be the reasons the spa asks more health parameters such as blood oxygen and heartbeat to Data4Help which again,before sharing these data with BeautySPA, will ask the authorization of the user.
%\subsubsection{Scenario 3}
%FastAndEasy, a car-sharing company, has a business model based on how much time his users rent one of his car in a week. After 5 years the society has to decide whether change this model with a new one based on the kilometers travelled in a week or keep going with the old one. In order to take the right decision FastAndEasy needs to know the GPS location of all of his users when they are driving his cars(so not 24 hours a day) for 3 months. Because of this is a temporary solution it will be a waste of money for the car-sharing company to implement his own system for picking up these data. Therefore FastAndEasy creates a third party account of Data4Help filling all the required third party information (see Table 1) and asks to the app to provide the GPS location only of his users(which can be identified by the fiscal code). This is a Single-user(Table 2) request therefore  Data4Help will ask directly to the specified users whether accept or deny it.\\ Indeed only for the users who accept the request the app will grant the access to their GPS location to the company which,  looking at the data collected after 3 months, can compute the possible earnings of the new model,compare them with the old ones and finally make a decision.


    %\subsubsection{Allow a Third Party to access at a speficy DataSet prior User Authorization}
    %	\subsubsection{Allow a Third Party to access at  group of anonymous DataSets made by at least 1000 individuals}
    %check whether the data fits with the request of the third party
    %	\subsubsection{Activation of AutomatedSOS must be granted in case of need}

  \subsection{External interface requirements}
    \subsubsection{User interfaces}
    \label{sec:userinterfaces}
    \subsubsection{Hardware interfaces}
    \subsubsection{Software interfaces}
    \subsubsection{Communication interfaces}

  \subsection{Functional requirements}

    \begin{description}
      \item[Account handling]
      \item[\texttt{R.A1}] The system shall allow users registration
      \item[\texttt{R.A2}] The system shall allow third party registration
      \item[\texttt{R.A3}] The system shall distinguish between user and third party accounts
      \item[\texttt{R.A4}] The system shall guardantee account uniqueness by not allowing two account to have the same email
      \item[\texttt{R.A6}] The system shall allow users and third parties to access their account (\textit{login}) only if they provide correct email and password
      \item[\texttt{R.A7}] The system shall allow users and third parties to exploit \texttt{Data4Help} and \texttt{AutomatedSOS} functionalities only if they are \textit{logged in} their account

      \item[Data encoding]
      \item[\texttt{R.D1}] The system shall encode data received through wearables'sensors and store it internally\footnote{for example the system can store data into an internal database or can save it using a cloud service; the important aspect is that data must be retrivable in the future}
      \item[\texttt{R.D2}] The system shall be able to retrive data previously stored on request
      \item[\texttt{R.D3}] The system shall not erase data once it is stored internally
      \item[\texttt{R.D4}] Once the system stored data, it shall allow access to that data only to the user account that was \textit{logged in} while the data was collected
      \item[\texttt{R.D5}] The system shall be able to share the stored data to more than one account
      \item[\texttt{R.D6}] The system shall be able to compose groups of data entries (\textit{aggregate data}) and anonymize them (every data entry in the group has no information about the providing user, such as fiscal code or email)

      \item[Interfaces]
      \item[\texttt{R.I1}] The system shall provide a registration form for users and third parties
      \item[\texttt{R.I2}] The system shall provide to the user an interface able to render data graphically, allowing filters like time interval or data type
      \item[\texttt{R.I3}] The system shall provide a request form to the third parties

      \item[Data sharing requests]
      \item[\texttt{R.R1}] The system shall allow third parties to ask for data sharing of a single user
      \item[\texttt{R.R2}] The system shall ask the user to accept or decline every single-user request of data sharing by third parties that has him/her as target user
      \item[\texttt{R.R3}] The system shall provide access to the target user data to the third party only if the user accepted the request of the third party, otherwise the system shall notify to the third party that the user declined the request
      \item[\texttt{R.R4}] The system shall allow third parties to ask for data sharing of \textit{aggregate data}
      \item[\texttt{R.R5}] The system shall be able to check if a request for \textit{aggregate data} by a third party can be properly anonymized (there shall be at least 1000 user data entries that fit the parameters of the request)
      \item[\texttt{R.R6}] The system shall provide access to the \textit{aggregate data} to the third party only if the the request can be properly anonymized, otherwise the system shall notify the third party that its request cannot be properly anonymized
      \item[\texttt{R.R7}] The system shall provide access to a third party to newly produced data if it fits the third party request
      % TODO add refernce to the request structure, because this requirement is a bit ambiguous

      \item[SOS calls]
      \item[\texttt{R.S1}] The system shall provide to the user an option to apply to \texttt{AutomatedSOS}
      \item[\texttt{R.S2}] If user applied to \texttt{AutomatedSOS}, the system shall monitor his/her parameters in real time by checking whether they are above or below the thresholds\footnote{TBD
      % TODO add reference
      }
      \item[\texttt{R.S3}] If user applied to \texttt{AutomatedSOS} and his/her health parameters are critical (above or below thresholds), the system shall send an emergency call to the SOS system
      \item[\texttt{R.S4}] When the system sends an emergency call, it shall provide to the SOS system API user's GPS location and user's health status through his/her critical parameters encoding

    \end{description}

%TODO Special requirements for 5 second instead of Constraint(section 2.4.3)

%\subsubsection{Sign up of User}
     \begin{table}[h!]
   \centering
    \begin{tabularx}{\linewidth}{|c|X|}
         \hline
         \textbf{Name} & Sign in of User\\
	\hline
	\textbf{Actors} & User \\
	\hline
	\textbf{Entry Condition} & Data4Help is downloaded on the smartphone \\
	\hline
	\textbf{Flow of Events} & 1.User click on "Create a new account" button in the homepage of the app.

					2.User fills all the mandatory fields("Age", "Name", "Surname", "Email", "Sex", "Password", "Fiscal 							Code").

					3.User insert in the fields "Shared Data" the data that he want to share with the app.

					4.User clicks on the "Confirm" button.

					5.The system create a DataSet for the User in his internal database.\\
	\hline
	\textbf{Exit Condition} & The User account has been successfully created. \\
	\hline
	\textbf{Exceptions} & 
					%1. The system find that User filled the field "Age" with 12.
					1.The system finds that User filled the field "Email"  with an email that is non-existent or that has 							already been associated with another user.

					2.The system finds that User filled the field "Fiscal Code" with an expression not long 16 								characters.

					3.The system finds that User filled the field "Password" with an expression shorter than 8 								characters.

					All exceptions are handled notifying the issue to the User and taking back the \textbf{Flow of 							Events} to the point 2.\\
        \hline
      \end{tabularx}        
      \end{table}

%\subsubsection{Log In}
 \begin{table}[h!]
   \centering
    \begin{tabularx}{\linewidth}{|c|X|}
         \hline
         \textbf{Name} & Log in of User\\
	\hline
	\textbf{Actors} & User \\
	\hline
	\textbf{Entry Condition} & User has already created an account\\
	\hline
	\textbf{Flow of Events} & 1.User opens the app on his smartphone.

					2.User insert his email in the field "Email" in the homepage of Data4Help.

					3.User insert his password in the field "Password" in the homepage of Data4Help.

					4.User Click on "Log In" button in the homepage of the app.\\
	\hline
	\textbf{Exit Condition} & User is successfully Logged in. \\
	\hline
	\textbf{Exceptions} & 1.The system doesn't find the email address inserted in the field "Email" in his internal DataBase. 
				
				In this case the \textbf{Flow of Events} has to be executed again from step 2. 

				2.The system doesn't find the password inserted in the field "Password" in his internal DataBase.

				 In this case the \textbf{Flow of Events} has to be executed again from step 3. 

				3.The system find both the email and the password inserted by User but these refers to 2 different users.               

				In this case the \textbf{Flow of Events} has to be executed again from step 2. \\ 
        \hline
      \end{tabularx}
      \end{table}

%\subsubsection{Accept Single-User request}
 \begin{table}[h!]
   \centering
    \begin{tabularx}{\linewidth}{|c|X|}
         \hline
         \textbf{Name} & Accept Single-User request\\
	\hline
	\textbf{Actors} & Third party and User \\
	\hline
	\textbf{Entry Condition} & Third party and User have already created respectively a third party and a user account and they 							are already logged in in Data4Help.\\
	\hline
	\textbf{Flow of Events} & 1.Third party clicks on the button "Single-User Request".

					2.Third party inserts the Fiscal Code of the User in the field "Fiscal Code" of the Single-User 								Request page.

					3.Third party lists to the system the data of the User which it would like to access filling the field 							"DataRequired" in the Single-User Request page.

					4.The system asks to the User whether he accepts or denies to share those specified data with the 					Third party.

					5.User answers clicking in the field "Yes" in his homepage.

					6.The system make available the data for the third party.\\
	\hline
	\textbf{Exit Condition} & Third party can exploit the data \\
	\hline
	\textbf{Exceptions} & 1.The system doesn't find the fiscal code inserted by third party in the field "Field Code" in his internal 				database.

				In this case the \textbf{Flow of Events} has to be executed again from step 2. 

				2.The system discovers that the User has denied the request therefore Data4Help notifies this 							information to the third party which won't have access to the required data.

				The third party can 	make a new Single-User request referred to a new User: in this case the 							\textbf{Flow of Events} has to executed again from the first step.\\

        \hline
      \end{tabularx}
      \end{table}

%
 \begin{table}[h!]
   \centering
    \begin{tabularx}{\linewidth}{|c|X|}
         \hline
         \textbf{Name} & Call an ambulance\\
	\hline
	\textbf{Actors} & User and SOS System\\
	\hline
	\textbf{Entry Condition} & User's health parameters exceedes thresholds\\
	\hline
	\textbf{Flow of Events} & 1.The system collects GPS location and health status parameters of the User from his internal 						database.

					2.The system makes an emergency call to SOS System by providing GPS location and health 							status feedback of the User.

					3.The SOS System accepts the call and dispatches the ambulance to the User location.

					4.The SOS System notifies the system when the ambulance arrives at User Location.\\
	\hline
	\textbf{Exit Condition} & The ambulance is taking care of the User. \\
	\hline
	\textbf{Exceptions} & \\
	\hline
	\textbf{Special Requirements} & 1.The system has to collect the GPS location,health parameters of the User and make the 							emergency call to SOS System within five seconds from the time the health parameters 							exceedes thresholds.\\
        \hline
      \end{tabularx}
      \end{table}

%\subsubsection{Accept Anonymous-Group request}
 \begin{table}[h!]
   \centering
    \begin{tabularx}{\linewidth}{|c|X|}
         \hline
         \textbf{Name} & Accept Anonymous-Group request\\
	\hline
	\textbf{Actors} & Third Party\\
	\hline
	\textbf{Entry Condition} & Third party has already created a third party account and is already logged in in Data4Help.\\
	\hline
	\textbf{Flow of Events} &
					1.Third party clicks on the button "Anonymous-Group Request".

					2.Third party lists to the system which data requires filling the field "Data Required" in the 							Anonymous-Group request page.

					3.The system looks for, in his internal database, the User's Datasets  which fit with the request 						made by the third party.

					4.The system aggregate the data found and anonymize them. 

					5.The system provides the anonymous data to the third party.\\
	\hline
	\textbf{Exit Condition} & Third party can exploit the data \\
	
	\hline
	\textbf{Exceptions} & 1.The data cannot be anonymized  by the system because it doesn't find at least 1000 							User Datasets which match with the third party request. Therefore the system cannot 							grant access to the data to the third party and has to notify her this information .

						The third party can make a new Anonymous-Group request asking for a new set of data: 							in this case the \textbf{Flow of Events} has to be executed again from the first step.\\

        \hline
      \end{tabularx}
      \end{table}



%\subsubsection{Show Data previously acquired}
 \begin{table}[h!]
   \centering
    \begin{tabularx}{\linewidth}{|c|X|}
         \hline
         \textbf{Name} & Show Data previously acquired\\
	\hline
	\textbf{Actors} & User\\
	\hline
	\textbf{Entry Condition} & User has already created a User account and is already logged in in Data4Help.\\
	\hline
	\textbf{Flow of Events} &
					1.User clicks on the button "Data Management".

					2.User decides which filter has to be applied to his DataSet filling the field "Filters" in the Data 						Management page.

					3.The system organize the User's DataSet following the User's instructions.

					4.The system show to the User these data thanks to a graphical interface. \\
	\hline
	\textbf{Exit Condition} & User can see his data in the specified way\\
	
	\hline
	\textbf{Exceptions} & 1 The system hasn't gathered any data yet because the user has just created an account. 

				Thanks to the graphical interface the system notifies the user that no data can be shown.

				Only after having collected some data the User can specified filters:the \textbf{Flow of Events} must be 				executed from the first step.\\

        \hline
      \end{tabularx}
      \end{table}
%everything referred to the system
  \subsection{Performance requirements}
\textbf{AutomatedSOS} implements a very important function:monitoring the human health; in this field having a lower time response can save a life.\\ Therefore, when the system detects than at least one of the health parameters exceedes the thresholds (user is in health danger),it must collect all the required data(GPS location and user's health status) and send them through an emergency call to an external service(SOS System) within 5 seconds.\\
Moreover Data4Help has to send every 1 second the required data to the subscribed third parties(both for Single-User requests and for Anonymous-Group requests) in order to grant them accurate enough data.
  \subsection{Design constraints}
    \subsubsection{Standards compliance}
    \subsubsection{Hardware limitations}
    \subsubsection{Any other constraint}
  \subsection{Software system attributes}
    \subsubsection{Reliability}
    \subsubsection{Availability}
    \subsubsection{Security}
The system requires the users(both Data4Help and AutomatedSOS) and third Parties(only in Data4Help), in order to exploit all the functionalities of the app and avoid unauthorized access, to authenticate through email and password.\\ These 2 credentials and all the Data collected by the system has to be securely stored in the internal database.\\ Moreover as  requirement \texttt{R.D6} says for privacy issues for all the Anonymous-Group requests the system must not allow the third parties to link data collected to the users that produced them.
    \subsubsection{Mantainability}
In case of failure the system must be able to recover within 1 second otherwise in case of user's health problem AutomatedSOS won't respect the constraint descibed in section 2.4.3 and therefore the user will be in a critical situation.\\
Moreove, Data4Help, as section 2.2.2 says, has to collect data through multiple physical wearables devices therefore 
Data type handling should be implemented by the system in a flexible and extensible way, because wearables technologies change rapidly and our software should adapt in the least invasive way possible.

 %Constraint vs Special requirement for 5 seconds
    \subsubsection{Portability}











